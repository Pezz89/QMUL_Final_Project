\documentclass[titlepage]{scrartcl}
\usepackage{enumitem}
\usepackage[british]{babel}
\usepackage[style=apa, backend=biber]{biblatex}
\DeclareLanguageMapping{british}{british-apa}
\usepackage{url}
\usepackage{float}
\usepackage[labelformat=empty]{caption}
\restylefloat{table}
\usepackage{perpage}
\MakePerPage{footnote}
\usepackage{abstract}
\usepackage{graphicx}
% Create hyperlinks in bibliography
\usepackage{hyperref}
\usepackage{amsmath}

\usepackage[T1]{fontenc}
\usepackage[utf8]{inputenc}
\usepackage{blindtext}
\setkomafont{disposition}{\normalfont\bfseries}

\graphicspath{{./resources/}}
\addbibresource{~/Documents/library.bib}

\newsavebox{\abstractbox}
\renewenvironment{abstract}
  {\begin{lrbox}{0}\begin{minipage}{\textwidth}
   \begin{center}\normalfont\sectfont\abstractname\end{center}\quotation}
  {\endquotation\end{minipage}\end{lrbox}%
   \global\setbox\abstractbox=\box0 }

\usepackage{etoolbox}
\makeatletter
\expandafter\patchcmd\csname\string\maketitle\endcsname
  {\vskip\z@\@plus3fill}
  {\vskip\z@\@plus2fill\box\abstractbox\vskip\z@\@plus1fill}
  {}{}
\makeatother

\DeclareCiteCommand{\citeyearpar}
    {}
    {\mkbibparens{\bibhyperref{\printdate}}}
    {\multicitedelim}
    {}

% MATLAB Code block stuff...
\usepackage{color}
\usepackage{listings}

\definecolor{dkgreen}{rgb}{0,0.6,0}
\definecolor{gray}{rgb}{0.5,0.5,0.5}

\lstset{language=Matlab,
   keywords={break,case,catch,continue,else,elseif,end,for,function,
      global,if,otherwise,persistent,return,switch,try,while},
   basicstyle=\ttfamily,
   keywordstyle=\color{blue},
   commentstyle=\color{gray},
   stringstyle=\color{dkgreen},
   numbers=left,
   numberstyle=\tiny\color{gray},
   stepnumber=1,
   numbersep=10pt,
   backgroundcolor=\color{white},
   tabsize=4,
   showspaces=false,
   showstringspaces=false}

\begin{document}
\title{ECS750P --- Final Project}
\subtitle{\LARGE{Extraction and Analysis of RRi from PCG Signals for the
Classification of Heart Abnormalities}}
\author{Sam Perry --- EC16039}

\maketitle

\section{Literature Review}
There are currently a wide variety of methods employed for the analysis and
classification of PCG signals. Current research focuses on a number of areas,
the most relevant of which are:
\begin{itemize}
    \item Algorithms for the segmentation of PCG data, aiming to extract the
        structure of the signal over time. This is a key stage in the analysis
        of PCG signals as relationships between the fundamental heart sounds
        (FHSs) form the basis for much of the further analysis performed on PCG
        data. A number of methods exist for the extraction of FHSs. Some rely on direct extraction of
        peaks in the time domain to determine the structure of a
        signal. These methods perform various transformation in order to
        accentuate the transient events.~\parencite{Groch1992, Liang1997}. However, these methods
        tend to suffer significantly from background noise and so perform
        poorly in sub-optimal conditions.\\
        Other methods rely on spectral representations to
        assist in the splitting of the FHSs, in particular using wavelet
        decomposition  ~\parencite{}. Machine learning
        algorithms have also been widely employed, such as k Nearest
        Neighbour~\parencite{} and Neural Networks~\parencite{} for
        predictions. Particular success has been observed in Springer's use of
        logistic regression and Hidden semi-Markov models~\citeyearpar{Springer2016}
    \item Methods for the extraction of statistical features from PCG data in
        order to create robust, meaningful representations of the data.
    \item Classification of signals for diagnostic purposes. The aim being to
        distinguish healthy signals from those with certain heart
        conditions/abnormality. Machine learning techniques are commonly used
        in order to distinguish between signals automatically, based on prior
        feature extraction. 
        it is noted in  that there is a lack of research into other machine
        learning techniques such as bayesian classification and
        SVMs~\citeyearpar{}.
\end{itemize}



        
A variety of machine learning techniques trained on these extracted
features. From this, a great deal of progress has been made in classifying a
variety of cardiac abnormalities such as. 

\printbibliography{}

\end{document}
