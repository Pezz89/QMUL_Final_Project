\documentclass[titlepage]{scrartcl}
\usepackage{enumitem}
\usepackage[british]{babel}
\usepackage[style=apa, backend=biber]{biblatex}
\DeclareLanguageMapping{british}{british-apa}
\usepackage{url}
\usepackage{float}
\usepackage[labelformat=empty]{caption}
\restylefloat{table}
\usepackage{perpage}
\MakePerPage{footnote}
\usepackage{abstract}
\usepackage{graphicx}
% Create hyperlinks in bibliography
\usepackage{hyperref}
\usepackage{amsmath}

\usepackage[T1]{fontenc}
\usepackage[utf8]{inputenc}
\usepackage{blindtext}
\setkomafont{disposition}{\normalfont\bfseries}

\graphicspath{{./resources/}}
\addbibresource{~/Documents/library.bib}

\newsavebox{\abstractbox}
\renewenvironment{abstract}
  {\begin{lrbox}{0}\begin{minipage}{\textwidth}
   \begin{center}\normalfont\sectfont\abstractname\end{center}\quotation}
  {\endquotation\end{minipage}\end{lrbox}%
   \global\setbox\abstractbox=\box0 }

\usepackage{etoolbox}
\makeatletter
\expandafter\patchcmd\csname\string\maketitle\endcsname
  {\vskip\z@\@plus3fill}
  {\vskip\z@\@plus2fill\box\abstractbox\vskip\z@\@plus1fill}
  {}{}
\makeatother

\DeclareCiteCommand{\citeyearpar}
    {}
    {\mkbibparens{\bibhyperref{\printdate}}}
    {\multicitedelim}
    {}

% MATLAB Code block stuff...
\usepackage{color}
\usepackage{listings}

\definecolor{dkgreen}{rgb}{0,0.6,0}
\definecolor{gray}{rgb}{0.5,0.5,0.5}

\lstset{language=Matlab,
   keywords={break,case,catch,continue,else,elseif,end,for,function,
      global,if,otherwise,persistent,return,switch,try,while},
   basicstyle=\ttfamily,
   keywordstyle=\color{blue},
   commentstyle=\color{gray},
   stringstyle=\color{dkgreen},
   numbers=left,
   numberstyle=\tiny\color{gray},
   stepnumber=1,
   numbersep=10pt,
   backgroundcolor=\color{white},
   tabsize=4,
   showspaces=false,
   showstringspaces=false}

\begin{document}
\title{ECS750P --- Final Project}
\subtitle{\LARGE{Extraction and Analysis of Statistical Features from PCG
Signals for the Classification of Heart Abnormalities}}
\author{Sam Perry --- EC16039}

\maketitle

\section{Literature Review}
There are currently a wide variety of methods are employed for the analysis and
classification of PCG signals. Current research focuses on a number of areas,
the most relevant of which are:
\begin{itemize}
    \item Algorithms for the segmentation of PCG data, aiming to extract the
        structure of the signal over time. This is a key stage in the analysis
        of PCG signals as relationships between the fundamental heart sounds
        (FHSs) form the basis for much of the further analysis performed on PCG
        data. A number of methods exist for the extraction of FHSs. Some rely
        on direct extraction of peaks in the time domain to determine the
        structure of a signal. These methods perform various transformation in
        order to accentuate the transient events with the intention of
        isolating them~\parencite{Groch1992, Liang1997}. However, these methods
        tend to suffer significantly from background noise and so perform
        poorly in sub-optimal conditions.\\
        Other methods rely on spectral representations to assist in the
        splitting of the FHSs, in particular using wavelet
        decomposition~\parencite{LiangHuiying1997, Vepa2008}. This allows for
        the separation of components based on their frequency content in
        place of or addition to their temporal characteristics.\\
        In addition,  Machine learning algorithms have been employed, such as k
        Nearest Neighbour~\parencite{Gupta2007} and Neural
        Networks~\parencite{Oskiper2002} to improve segment classification.
        More recently, particular success has been observed in Springer's use
        of logistic regression and Hidden semi-Markov
        models~\citeyearpar{Springer2016}.
    \item Signal Pre-processing?
        Removal of ectopic beats in RR estimation~\parencite{Dash2009}

    \item A wide variety of methods exist for the extraction of statistical
        features from PCG data. These features are used for the creation of
        robust, meaningful representations of the data.\\
        The use of spectral representations for PCG data are prominent in the
        literature. The ability to separate activity across the frequency
        spectrum reveals patterns that may not be attainable by analysing the
        time domain signal alone.\\
        Due to the need for low frequency analysis and the high noise levels
        found in PCG signals, it has been found that the traditional FFT
        method for extracting spectral information may not be
        suitable~\parencite{Akay1990}. For this reason, parametric methods for
        spectral estimation have been a popular choice for extraction of such information. 
        Methods such as AR, ARMA, AR-HOS and MUSIC have been shown to provide spectral
        representations suitable for analysis and classification of heart
        sound~\parencite{Ergen2001, Schmidt2015}.\\
        Other methods such as Wavelet Decomposition and MFCCs have also been
        successfully  employed for extracting spectral data for purposes such
        as heart valve disease identification and heart murmur
        detection~\parencite{Quiceno-Manrique2010a, Maglogiannis2009}.\\
        
        In addition to direct analysis on the signal, the ability to segment
        and extract RR values from the signal allows for their statistical
        analysis, both in the time and frequency domain, for use as features.\\
        Dash et al. use a number of time-based statistical analysis on the RR
        time series for the detection of atrial fibrilation. Statistical
        analyses such as RMSSD, Shannon Entropy and Turning-point Ratio are
        used as feature vectors for classification of
        signals~\parencite{Dash2009}.  A similar approach is used by Yaghouby
        et al. for the generalized classification of heart abnormality. Here,
        other features such as HR Mean, Standard deviation, pNN50 and
        Triangular Index are used for classification with promising
        results~\parencite{Yaghouby2009}.

        Frequency domain analysis of RR values can also be considered by
        calculating the PSD of the RR values via similar methods for spectral
        analysis as with the direct signal.

        RR Frequency Domain Features
        
        RR Time-frequency domain features VFCDM~\parencite{Dash2009}
        RR Non-linear features
        ~\parencite{Yaghouby2009}

        

    \item Classification of signals for diagnostic purposes. The aim being to
        distinguish healthy signals from those with certain heart
        conditions/abnormality. Machine learning techniques are commonly used
        in order to distinguish between signals automatically, based on prior
        feature extraction. 
        it is noted in  that there is a lack of research into other machine
        learning techniques such as bayesian classification and
        SVMs~\citeyearpar{}.
\end{itemize}



        
A variety of machine learning techniques trained on these extracted
features. From this, a great deal of progress has been made in classifying a
variety of cardiac abnormalities such as. 

\pagebreak{}
\printbibliography{}

\end{document}
