\documentclass[titlepage, 12pt]{scrartcl} \usepackage{enumitem}

\usepackage[british]{babel}
\usepackage[style=apa, backend=biber]{biblatex}
\DeclareLanguageMapping{british}{british-apa}
\usepackage{url}
\usepackage{float}
\usepackage{caption}
%\restylefloat{table}
\usepackage[table]{xcolor}
\usepackage{multirow}
\usepackage{perpage}
\MakePerPage{footnote}
\usepackage{abstract}
\usepackage{graphicx}
\usepackage{setspace}
% Create hyperlinks in bibliography
\usepackage{hyperref}
\usepackage{amsmath}
\usepackage{booktabs}
\usepackage{tabulary}

\usepackage[pass]{geometry}
\usepackage{pdflscape}
\usepackage{graphicx}

\usepackage[T1]{fontenc}
\usepackage[utf8]{inputenc}
\usepackage{blindtext}
\setkomafont{disposition}{\normalfont\bfseries}

\usepackage{etoolbox}

\usepackage{titlesec}
\setcounter{secnumdepth}{4}

\titleformat{\paragraph}
{\normalfont\normalsize\bfseries}{\theparagraph}{1em}{}
\titlespacing*{\paragraph}
{0pt}{3.25ex plus 1ex minus .2ex}{1.5ex plus .2ex}

\graphicspath{{./resources/}}
\addbibresource{~/Documents/library.bib}

\DeclareMathOperator*{\argmax}{arg\,max}
\DeclareMathOperator*{\argmin}{arg\,min}
% Fix for medeley's rubbish underscore handeling in generated bib files
\DeclareSourcemap{
    \maps{
        \map{ % Replaces '{\_}', '{_}' or '\_' with just '_'
            \step[fieldsource=url,
                  match=\regexp{\{\\\_\}|\{\_\}|\\\_},
                  replace=\regexp{\_}]
        }
        \map{ % Replaces '{'$\sim$'}', '$\sim$' or '{~}' with just '~'
            \step[fieldsource=url,
                  match=\regexp{\{\$\\sim\$\}|\{\~\}|\$\\sim\$},
                  replace=\regexp{\~}]
        }
        \map{ % Replaces '{\_}', '{_}' or '\_' with just '_'
            \step[fieldsource=url,
                  match=\regexp{\{\\\#\}|\{\#\}|\\\#},
                  replace=\regexp{\#}]
        }
    }
}

%\newsavebox{\abstractbox}
%\renewenvironment{abstract}
%  {\begin{lrbox}{0}\begin{minipage}{\textwidth}
%   \begin{center}\normalfont\sectfont\abstractname\end{center}\quotation}
%  {\endquotation\end{minipage}\end{lrbox}%
%   \global\setbox\abstractbox=\box0 }

%\makeatletter
%\expandafter\patchcmd\csname\string\maketitle\endcsname
%  {\vskip\z@\@plus3fill}
%  {\vskip\z@\@plus2fill\box\abstractbox\vskip\z@\@plus1fill}
%  {}{}
%\makeatother

\newcommand{\dtoprule}{\specialrule{1pt}{0pt}{1.4pt}%
            \specialrule{1pt}{0pt}{\belowrulesep}%
            }
\newcommand{\dbottomrule}{\specialrule{1pt}{0pt}{1.4pt}%
            \specialrule{1pt}{0pt}{\belowrulesep}%
            }

\DeclareCiteCommand{\citeyearpar}
    {}
    {\mkbibparens{\bibhyperref{\printdate}}}
    {\multicitedelim}
    {}

% MATLAB Code block stuff...
\usepackage{color}
\usepackage{listings}

\definecolor{dkgreen}{rgb}{0,0.6,0}
\definecolor{gray}{rgb}{0.5,0.5,0.5}

\lstset{language=Matlab,
   keywords={break,case,catch,continue,else,elseif,end,for,function,
      global,if,otherwise,persistent,return,switch,try,while},
   basicstyle=\ttfamily,
   keywordstyle=\color{blue},
   commentstyle=\color{gray},
   stringstyle=\color{dkgreen},
   numbers=left,
   numberstyle=\tiny\color{gray},
   stepnumber=1,
   numbersep=10pt,
   backgroundcolor=\color{white},
   tabsize=4,
   showspaces=false,
   showstringspaces=false}
\usepackage[shortcuts]{extdash}

\definecolor{codegreen}{rgb}{0,0.6,0}
\definecolor{codegray}{rgb}{0.5,0.5,0.5}
\definecolor{codepurple}{rgb}{0.58,0,0.82}
\definecolor{backcolour}{rgb}{0.95,0.95,0.92}

\lstdefinestyle{mystyle}{
   keywords={},
    numberstyle=\tiny,
    basicstyle=\footnotesize,
    breakatwhitespace=false,
    breaklines=true,
    captionpos=b,
    keepspaces=true,
    numbers=left,
    numbersep=5pt,
    showspaces=false,
    showstringspaces=false,
    showtabs=false,
    tabsize=2
}
\lstset{style=mystyle}

\begin{document}
\newgeometry{lmargin=1.5cm}
\begin{titlepage}

    \begingroup

    \setlength{\tabcolsep}{1.5cm}

    \begin{tabular}[c]{p{0.30\textwidth} | p{0.4\textwidth}}

    {\vspace{1.2cm} \Large School of Electronic Engineering and Computer Science \par}
    &
    {\vspace{1.2cm} \large Sound and Music Computing \newline Project Report \the\year \par}\\

    & {\vspace{0.5cm} \Large \textbf{Extraction of Audio Features from PCG Signals for the
Classification of Heart Abnormalities} \par}\\

    \vspace{0.4\textheight}
    \includegraphics[width=5cm]{qmul_logo}
    &
    {\vspace{1cm} \large \textbf{Samuel Perry}}\\

    &
    \multicolumn{1}{|r}{August \the\year}

    \end{tabular}

    \endgroup

\end{titlepage}
\restoregeometry

\doublespacing
\begin{abstract}
   Things and stuff and words...
\end{abstract}

\renewcommand{\abstractname}{Acknowledgements}
\begin{abstract}
I'd like to thank anyone and everyone...
\end{abstract}

\tableofcontents
\newpage

\section{Introduction}
Cardiovascular diseases are the most prevalent cause of death in Europe,
accounting for 37.5\% of all deaths in 2013~\parencite{Eurostat2016}.
Traditionally, cardiac auscultation has been performed manually using a standard
stethoscope, with the aim of detecting heart defects aurally. This has been a
fundamental method for detecting heart valve disorders for over a century.
However, auscultation is a skill that requires training and can only usually be
performed by a medical professional, such as a GP. As a result, manual
auscultation is significantly susceptible to human error~\parencite{Hanna2002}.
Automation of this method using technology may be provide a solution, and
recent research has shown promise in this area. A large amount of research has
focused on analysis of Electrocardiogram (ECG) signals.  Although useful for
detecting pathologies, ECG equipment is expensive and requires a trained
professional for use. Therefore it is not currently feasible for developing
countries and rural areas where there may be few physicians available. A
comparatively affordable and non-invasive alternative is the Phonocardiogram
(PCG)~\parencite[p.130]{Reed2004}. Typically recorded using an electronic
stethoscope, a PCG signal is a recording of sound made as the heart contracts,
analogous to the sound heard by physicians when performing cardiac auscultation
manually. Automated auscultation could provide an initial diagnosis for heart
defects without the need for a trained medical professional. This would allow
relatively cheap equipment to analyse a patient's heart sound, and
automatically recommend further inspection based on analysis.  By providing
earlier diagnosis of conditions that may have otherwise been overlooked, this
technology could have a significant impact on reducing mortality rates as a
result of heart conditions.
% TODO: Write brief overview of history of PCG signal analysis
% TODO: Explain fundamental heart sounds

\section{Related Work}
There are currently a wide variety of methods employed for the analysis and
classification of PCG signals. Current methods can typically be divided into 3
areas, each of which are combined to create a full classification system. These
areas are: signal preprocessing, signal segmentation, and feature
extraction/classification. The performance and evaluation of complete systems
are also discussed in section~\ref{Classification}
% TODO: Make flow diagram of 3 stages


\subsection{Signal Preprocessing}
There are a large number of factors that lead to variation in quality of PCG
recordings: stethoscope type, make and model, its microphone/sensors, the
position used to record (i.e.\ lower left sternal border, apex, pulmonic area,
aortic area), built in filters/signal processing used by the stethoscope (i.e.\
noise filters, anti-tremor filters), medication that a patient may be taking,
as well as many other aspects that may influence the recorded
signal~\parencite[p.4]{Pavlopoulos2004}. This presents a significant issue when
attempting to analyse and compare a database of signals, as variations in
recordings and artefacts caused by factors other than heart sounds will most
likely interfere with analysis and comparison methods. To account for this,
pre-processing methods are widely used, aiming to standardise a database. This
is also used as a way to accentuate features of the data that are expected to
be relevant for classification.\\

A common method employed is the use of decimation and static filters to remove
unwanted spectral content that is most likely noise~\parencite{Liang1997a,
Homsi2016, Springer2016, Gupta2007}. This helps reduce higher frequency noise
such as speech, microphone movement, breathing and other interference caused
externally. Signals are commonly downsampled to around 1--4KHz, with
anti-aliasing filter specifications varying across the literature. Generally,
highpass chebychev or butterworth filters are favoured with cutoff frequencies
ranging from 400--750Hz.\\

In addition, many methods decompose the filtered signal using wavelet based
methods, such as the discrete wavelet transform (DWT)~\parencite{Liang1997a,
Pavlopoulos2004}, continuous wavelet transform (CWT)~\parencite{Langley2016} or
wavelet package decomposition (WPD)~\parencite{Liang1998}, are commonly used to
separate components of a signal based on their spectral content.
Wavelet transforms are popular as, unlike Fourier transforms, they are well
localised in both the time and frequency domain. This allows for the analysis
of PCG signals across multiple frequency bands whilst maintaining transient
temporal events in the resulting decomposition~\parencite[p.93]{Ari2008}.
This may be used for analysis of transient events such as murmurs, that may
consist of higher frequency components than normal heart sounds.

\subsection{Signal Segmentation}\label{Segmentation}
Algorithms for the segmentation of PCG data aim to extract the structure of the
signal over time. This is a key stage in the analysis of PCG signals, as the
structure of the signal and relationships between the fundamental heart sounds
(FHSs) form the basis for much of further analysis performed on PCG data.\\

% TODO: insert segmented graph of PCG cycle

A number of methods exist for the extraction of FHSs. Traditional methods rely
on direct extraction of peaks from amplitude envelopes in the time domain to
determine the structure of a signal.  These methods perform various
processing/transformations in order to accentuate the transient events with the
intention of isolating them.\\
Early work in this area by Liang et.\ al described a method using the popular
Shannon energy envelope, achieving good accuracy across 37 recordings of
children~\parencite{Liang1997b}. The algorithm aimed to segment the data by
first extracting the envelope, then applying adaptive rule based thresholds, to
determine peaks corresponding to segmentation points. When comparing results to
hand-annotated ground truth data, the system achieved a reported accuracy score
of 84\%. However, due to the small sample size, and potential lack of noise in
the database used, this may not translate well to a larger database recorded in
sub-optimal conditions.\\
More recent methods used spectral representations to assist in the splitting of
the FHSs, in particular using wavelet decomposition. These methods tend to
perform more robustly on signals of varying conditions.\\
Building on previous work, Liang et.\ al presented an improved method, using the
discrete wavelet transform to decompose and reconstruct the signal into 7
distinct frequency bands~\parencite{Liang1997a}. Applying a similar method
of envelope extraction and peak picking to each frequency band, the best
estimate of all frequency bands is then chosen as the final result. Criterion
for this choice is based on the number of S1s and S2s detected, and the number
of artefacts discarded for each frequency band. This method achieved an
improved accuracy of 93\% across a larger database of 77 recordings. This
suggests that the algorithm is as robust if not more so than previous work by
Liang et al.\\

Vepa et al.\ proposed a wavelet decomposition based method that uses a
combination of simplicity and envelope features~\parencite{Vepa2008}. This
approach attempts to improve robustness when analysing signals of varying
quality by using multiple complimentary features. This allows the method to base
decisions on a variety of statistical properties. Evaluating the algorithm on a
collection of 160 heart cycles from a variety of sources, a reported accuracy
of 84\% was achieved.\\

More recently, a variety of machine learning methods have been implemented with
reasonable success. Gupta et al.\ presented a method that applies $k$-means
clustering to replace standard threshold-based methods for determining peak
classification in a segmentation algorithm based on standard
envelopes~\parencite{Gupta2007}. This achieved a reported accuracy of 90.29\%.
Due to the envelope-based method for feature extraction, this method is still
suceptible to noise and artefacts that occur within the frequency bands of the
heart sounds.\\

Sepehri et.\ al proposed a method that combines neural networks with Power
Spectral Density (PSD) estimates~\parencite{Sepehri2010}.  By exploiting the
periodic nature of S1 and S2 heart sounds, combined with their narrow frequency
range, a neural network is trained to separate these sounds from other events,
such as noise and murmurs. This method achieved a reported 93.6\% accuracy on a
significantly larger database than previous methods detailed.\\

The most significant successes in segmentation algorithms have been observed through use
of probabilistic models such as Hidden Markov Models (HMMs). Early research
using these models by Ricke et.\ al utilised embedded HMMs to model the 4
states of the PCG and their transitions~\parencite{Ricke2005}. MFCCs and
Shannon Energy were used as feature vectors for the models. Results of
98\% accuracy were reported, although this was tested on only a small database
of signals.\\
Gill et.\ al achieved similar results, most notably with specific consideration
for the duration of each state in the HMM~\parencite{Gill2005}. This is
handled through the extraction of 6 duration features based primarily on peaks.
These features form vectors for training the HMM. Results of 98.6\%
sensitivity, 96.9\% positive predictivity for S1 sounds and 98.3\% sensitivity,
96.5\% positive predictivity for S2 sounds is reported.
The issue of state duration was further addressed by Schmidt et.\ al through use
of a duration-dependent hidden Markov (DHMM)~\parencite{Schmidt2015}. The
DHMM is a modified HMM that considers the duration of the current state when
calculating the probability of transition to another state. This modification
scored a reported sensitivity of 98.8\% and a positive predictivity of
98.6\%.\\
Building on previous work using HMMs, Springer et al.\ presented a segmentation
algorithm by using hidden semi-markov models (HSMMs) in combination with
logistic regression~\parencite{Springer2016}. Use of Hidden semi markov model
allows for a priori information on the duration of the current state to be used
in probability calculation of the subsequent state. In this case, the knowlege
that there is an upper and lower limit on the duration of each component is
used in calculation of transition probabilities.  A modified viterbi algorithm
is then used to calculate the most likely set of transitions based on observed
features. Logistic regression is used to improve discrimination between state
features when compared to discriminatory methods used by previous work.
Performance was evaluated on a significantly larger database than previous
methods and achieved a reported accuracy of $95.63\% \pm 0.85\%$. Due to it's
rigorous evaluation and high accuracy, this method is currently considered the
state-of-the-art for PCG signal segmentation.\\

Table~\ref{SegmentationTable} provides a brief overview of significant research
into PCG segmentation. For a more complete summary of the current state of PCG
segmentation, please refer to Liu et.\ al~\parencite{Liu2016}

\newgeometry{margin=1cm} % modify this if you need even more space
\begin{landscape}
\begin{table}[htbp]
    \captionof{table}{Summary of Segmentation Algorithms} \label{SegmentationTable}
\scriptsize
%\centering
\rowcolors{1}{gray!15}{white}
\doublespacing
\begin{tabulary}{\linewidth}{LLLLL}
\dtoprule
Author                 & Method                                                                                         & databases                                                                                       & \mbox{Reported} Results         & Notes                                                                                            \\ \bottomrule
Springer et.\ al \citeyearpar{Springer2016} & HSMM, Logistic regression                                                                       & 10,172s of recordings from 112 patients. 12,181 first and 11,627 second heart sounds. & $95.63\pm0.85\%$                             & Supervised algorithm.                                                                                                                                                            \\
Huiying et.\ al \citeyearpar{Liang1997b} & Normalised average Shannon energy envelope, peak picking                                        & 37 recordings, 14 pathological murmurs and 23 physiological murmurs. 515 cycles       & $91.03\%\;Ac$                                          & Unsupervised Algorithm.  database consists entirely of child recording. Optimized on full database                                                                                 \\
Vepa et.\ al \citeyearpar{Vepa2008}     & Wavelet decomposition, energy and simplicity measurement                                       & 160 heart cycles collected from a variety of sources (training CDs, web resources)    & $84\%\;Ac$                                             & Unsupervised Algorithm, Optimized on full database                                                                                                                                \\
Sun et.\ al \citeyearpar{Sun2014}             & Viola integral envelope extraction, short-time modified Hilbert transform, peak picking        & 6949s of recordings, from 121 patients                                                & $97.37\%\;Ac$                                          & Supervised algorithm. Tolerance for segmentation accuracy not specified                                                                                                          \\
Sepehri et.\ al \citeyearpar{Sepehri2010}        & Spectral density estimation, auto-regressive parameters, multi-layer perceptron neural network & 120 recording, from 60 patients                                                       & $93.6\%\;Ac$                                           & Supervised algorithm                                                                                                                                                             \\
Ricke et.\ al \citeyearpar{Ricke2005}    & Shannon energy (and related features), HMM                                                     & 9 recordings, from 9 patients                                                         & $98\%\;Ac$                                             & Supervised algorithm                                                                                                                                                             \\
Schmidt et.\ al \citeyearpar{Schmidt2015}  & DHMM, Auto-correlation duration features, Homomorphic envelogram                               & 113 recordings, from 113 patients. 8s per recording. 15 abnormal recordings           & $98.8\;Se,\;98.6\;P_+$ on test set                         & All data recorded ``lateral to the sternum in the fourth intercostal space on the left side''. Mix of noisy and clean recordings. 40 recording used for training, 73 for testing \\
Gill et.\ al \citeyearpar{Gill2005}         & Homomorphic envelogram, Embedded HMMs                                                          & 44 recording, 17 subjects. 30-60s per recording                                       & $98.6\%\;Ac, 96.9\;P_+$ for S1. $98.3\;Ac,\;96.5\;P_+$ for S2 & Recording taken in sub-optimal environments (noisy hospitals, offices etc...)                                                                                                    \\
Gupta et.\ al \citeyearpar{Gupta2007}    & Homomorphic filtering, $k$-means clustering                                                       & 41 patients, 340 heart cycles. 110 normal,  124 systolic murmur, 106 diastolic murmur & $90.29\%\;Ac$                                          & Unsupervised Algorithm.                                                                                                                                                          \\ \hline
\dbottomrule\\
% TODO: Add footnote explanation for Ac = Accuracy
% TODO: Add citeyearpar references to authors
\end{tabulary}
\end{table}
\end{landscape}
\restoregeometry


\doublespacing

\subsection{Feature extraction/Classification models}\label{Classification}

A wide variety of methods exist for the extraction of statistical features and
classification of PCG data. Most notably, the recent Physionet/Computing in
Cardiology (CinC) Challenge 2016 has prompted the development of a range of
methods that have improved the quality of abnormality classification in noisy
signals.  The challenge was assembled to provide researchers with a large
database of normal/pathological PCG signals of varying quality. This enabled
the development of algorithms that could be evaluated on a significant
database, in order to determine performance across a range of conditions/signal
qualities~\parencite{Clifford2016}. This section first details significant work
produced prior to the challenge, then highlights key works produced for the
challenge to outline the breadth of methods for robust heart sound analysis.

\subsubsection{Work prior to the Physionet challenge}
Work prior to the Physionet challenge was conducted predominantly with the aim
of classifying specific heart conditions. Until recently, little research had
been produced with regards to general abnormality detection, with many projects
choosing to focus on specific conditions such as murmurs, atrial fibrillation,
flutter, and heart valve disease. This section outlines some key research
into these areas, alongside initial research into general abnormality
detection.\\

Reed et al.\ implemented a simple general classification algorithm using
artificial neural networks (ANNs) and wavelet
decomposition~\parencite{Reed2004}. As initial work into this field,
preprocessing such as segmentation is not performed and features remain
relatively simple when compared to more recent methods. Also, due to the
comparitively small sample size used for training (1 patient per abnormality, 4
cycles per patient), a reported accuracy of 100\% would have a strong
possibility of generalising poorly. This does however, serve as an early
example of limited success in general heart sound classification.\\

Maglogiannis et al.\ presented a classifier for discrimination of heart valve
disease from regular heart sounds using an SVM (Support Vector Machine)
classifier~\parencite{Maglogiannis2009}.  Roughly 100 features were extracted
from the signal, based on direct analysis of each heart cycle component (S1,
Systole, S2, Diastole) and the average shannon energy envelope of these
components.  A database of 198 heart sounds was curated for the project,
acquired from 8 sources, such as medical CDs and pre-existing databases.  An
accuracy of 91.43\% was reported using 10-fold stratified cross-validation.  In
addition, the project aimed to classify individual abnormalities in a 3 step
process, by distinguishing between systolic or diastolic murmurs, and then
distinguishing between aortic or mitral diseases. The classifier achieved
accuracy between 90-97\% for these classifications. This approach demonstrates
the potential for a system to accurately distinguish between normal and
abnormal heart sounds in a generalisable way, given carefully selected
features.\\

Ari et.\ al also proposed an SVM based method for abnormality
classification~\parencite{Ari2010}. A modified Least-squares SVM (LSSVM) is
used in order to improve separability between normal and abnormal datapoints
during training. 32 wavelet based features from previous literature are used as
feature vectors for a modified LSSVM, un-modified LSSVM and a standard SVM.
Comparison of the system shows that the proposed technique performs
significantly better on all test sets with an accuracy of between 86\% and
100\%, dependent on database. This research highlights the importance of
choosing an appropriate classification method for achieving accurate results.\\

Quiceno-Manrique et.\ al demonstrate the use of various time frequency
representations (TFR) such as short-time fourier transform, wavelet transforms,
Wigner-Ville distribution etc\ldots, with a $k$-nearest neighbour classifier
(k-NN) for systolic murmur detection~\parencite{Quiceno-Manrique2010a}. This
work highlights the effectiveness of alternative TFRs to traditional fourier
methods. This method also employs Principle Component Analysis (PCA), ie.\ the
mapping of a high dimensional feature space to a lower dimension, in order to
improve computational performance. Features were evaluated using a database
of of 22 patients, 6 of which were labeled as having a systolic murmur. The
highest reported accuracy was achieved using MFCCs as the primary feature
vector achieving a 98\% accuracy on 10-fold cross validation.\\

Schmidt et.\ al aimed to find features that could be used for classification of
coronary artery disease through detection of small
murmurs~\parencite{Schmidt2015}. A large number of features are
calculated to provide vectors for classification; Parametric spectral features
such as ARMA are used, alongside instantaneous frequency and octave power
measurements. Complexity features, such as sample entropy and simplicity, are
also calculated in an attempt to exploit the likely stochastic nature of
murmurs when compared to normal heart sounds.  Given the large number of
features calculated, PCA is used to retain only the most relevant information.
Quadratic discriminant analysis (QDA) is then used as a classifier to provide a
final accuracy score of 73\%.\\

An overview of significant research prior to the Physionet challenge is
provided in table~\ref{SumPrior}. It is also noted that none of the databases
used for prior research are publicly available.


\newgeometry{margin=1cm} % modify this if you need even more space
\begin{landscape}
\begin{table}[htbp]
    \captionof{table}{Summary of research prior to the Physionet Challenge 2016}\label{PriorWorkTable}
\scriptsize
%\centering
\rowcolors{1}{gray!15}{white}
\label{SumPrior}
\doublespacing
\begin{tabulary}{\linewidth}{LLLLLL}
\dtoprule
Author                   & Pre-processing/segmentation                                                                                                               & Features                                                                                                        & Classification Method & Database                                                                                                                 & Reported Accuracy                                  \\ \hline
Maglogiannis et.~al \citeyearpar{Maglogiannis2009}     & Wavelet decomposition, Shannon energy peak picking                                                                                        & Features derived from wavelet decomposition and PCG segmentations                                               & SVM                   & 198 recordings, 38 normal, 41 AS systolic murmur, 43 MR systolic murmur, 38 AR diastolic murmur, 38 MS diastolic murmur & $91.43\%\;Ac$                                      \\
Ari et.~al \citeyearpar{Ari2010}              & Amplitude envelope peak picking~\parencite{Ari2007}                                                                                       & Wavelet based features                                                                                          & LSSVM                 & 64 patients, 64 recordings, 512 cycles                                                                                  & $88.750-100\%\;Ac$ (dependant on abnormality type) \\
Quiceno-Manrique et.~al \citeyearpar{Quiceno-Manrique2010a}& Downsampled to 4KHz, Normalised to maximum of signal, ECG assisted QRS complex detection algorithm used for segmentation                  & Spectral features derived from STFT, Wavelet decomposition and quadratic energy distributions                   & $k$-NN                & 22 patients, 16 normal, 6 abnormal, 8 recordings (12s) per patient                                                      & $98\%\;Ac$                                         \\
Schmidt et.~al \citeyearpar{Schmidt2015}          & Signal filtered into frequency bands, Segmented by HMM based method+hand corrected, removal of high variance sub-segments to remove noise & Parametric spectral features (AR, ARMA and Music), Instantaneous frequency and amplitude, Power in octave bands & QDA                   & 435 Recordings, 133 patients, 70 normal, 63 abnormal                                                                    & $73\%\;Ac$                                         \\
Reed et.~al \citeyearpar{Reed2004}             & ---                                                                                                                                       & Wavelet decomposition coefficients, Manual feature reduction                                                    & ANN                   & 5 patients, 4 cycles per patient                                                                                        & $100\%\;Ac$                                        \\
\dbottomrule\\
% TODO: Add footnote explanation for Ac = Accuracy
% TODO: Add citeyearpar references to authors
\end{tabulary}
\end{table}
\end{landscape}
\restoregeometry

\subsubsection{Physionet challenge entries}\label{ChallengeEnt}
\doublespacing
The 2016 Physionet/CinC Challenge aimed to encourage development of heart
abnormality detection algorithms by providing a large open database of PCG
signal recordings, sourced from a variety of both clinical and non-clinical
environments. (Further details on the database can be found in
section~\ref{Database}. The complete specification is presented by Liu et.\
al~\parencite{Liu2016}). In addition, participants were provided with a
state-of-the-art heart sound segmentation algorithm, as proposed by Springer
et.\ al in Section~\ref{Segmentation}. Participants were then tasked with the
creation of a classification algorithm that could robustly discriminate between
healthy and unhealthy heart sound samples. The challenge recieved 348 entries
in total, each of which was scored on a hidden test database
using a Modified accuracy measure ($MAcc$) as defined by Clifford et.
al~\parencite{Clifford2016}:
\begin{table}[htbp]
\centering
\caption{Output Classification}
\label{OutputClassification}
\doublespacing
\begin{tabular}{llccc}
\hline
                              &                 & \multicolumn{3}{c}{Algorithm's Output}                                                    \\ \hline
                              &                 & \multicolumn{1}{l}{Normal} & \multicolumn{1}{l}{Uncertain} & \multicolumn{1}{l}{Abnormal} \\
\multirow{4}{*}{Ground Truth} & Normal, clean   & $Nn_1$                     & $Nq_1$                        & $Na_1$                       \\
                              & Normal, noisy   & $Nn_2$                     & $Nq_2$                        & $Na_2$                       \\
                              & Abnormal, clean & $An_1$                     & $Aq_1$                        & $Aa_1$                       \\
                              & Abnormal, noisy & $An_2$                     & $Aq_2$                        & $Aa_2$                       \\ \hline
\end{tabular}
\end{table}

\doublespacing

Weights are calculated as:
\begin{table}[H]
\centering
\doublespacing
\begin{tabular}{ll}
$Wa_1 = \frac{\text{Clean abnormal recordings}}{\text{Total abnormal recordings}}$ & $Wa_2 = \frac{\text{Noisy abnormal recordings}}{\text{Total abnormal recordings}}$ \\
$Wn_1 = \frac{\text{Clean normal recordings}}{\text{Total normal recordings}}$     & $Wn_2 = \frac{\text{Noisy normal recordings}}{\text{Total normal recordings}}$
\end{tabular}
\end{table}

Modified sensitivity ($Se$), specificity ($Sp$) and overall accuracy ($MAcc$) are then calculated as:

\begin{align*}
    &Se=Wa_1\frac{Aa_1}{Aa_1+Aq_1+An_1}+Wa_2\frac{Aa_2+Aq_2}{Aa_2+Aq_2+An_2} \\
    &Sp=Wn_1\frac{Nn_1}{Na_1+Nq_1+Nn_1}+Wn_2\frac{Nn_2+Nq_2}{Na_2+Nq_2+Nn_2} \\
    &MAcc=\frac{Se+Sp}{2}
\end{align*}

This section summarises some of the key works presented for the challenge,
including some of the most accurate models, and a baseline classifier
provided to participants as a starting point.\\

This baseline classifier was provided in order to demonstrate the basic
structure of systems expected for entries~\parencite{Liu2016}. The classifier
extracted a selection of 20 basic features, primarily focused on relative
timings and amplitudes of heart sounds.  A binary logistic regression model was
chosen for classification. From the 20 extracted features, 13 were selected
based on their statistical significance, measured using forward liklihood ratio
selection. The system achieved a reported score of 66\% on the test set, giving
a baseline score for participants to build on.  In addition, the system was
trained using leave-one-out cross validation. By removing a single training
database on each fold, the generalisation of the algorithm could then be
evaluated, training on all other databases. Results showed that performance
decreased significantly when training via this method, giving an average
accuracy of 59\%, with training database $b$ scoring as low as 47\%.  This
could suggest that individual databases in the database are not sufficiently
represented by other databases, or that features do not model abnormalities
sufficiently.\\

Homsi et.\ al proposed a system that utilised 131 time domain, STFT-based and
wavelet-based features which, when combined with nested ensemble classifiers,
produced an accuracy score of 84.48\%~\parencite{Homsi2017}. This algorithm
combines many commonly-used features in previous PCG related literature such as
wavelet decomposition based features, MFCCs and Shannon Energy. The system also
uses a total of 40 classifiers, 20 for signals labeled to be `standard' and 20
for thos labeled as `atypical'. A mixture of Random Forrest, LogitBoost and
Cost-Sensitive Classifiers (CSC) are used to classify signals in parallel.
Final results are combined using a rule-based decision, designed through manual
experimentation.\\
% TODO: Read into accuracy results for this method more closely

Potes et al.\ presented a similar approach to that of Homsi et
al.~\parencite{Potes2016}. 124 similar TFR features were extracted and used as
vectors for an AdaBoost classifier. This classifier was then combined with a
deep learning approach using a Convolutional Neural Network (CNN) classifier.
The signal was decomposed into 4 frequency bands and segmented, to provide
input to the CNN. Results from both AdaBoost and CNN classifiers were then
combined using a set descision rule.  This method produced the highest score on
the test set for the challenge at 86.02\%.\\

Zabihi et al.\ took an alternative approach by choosing not to segment PCG data
in the pre-processing stage~\parencite{Zabihi2016}. This was with the intention
of reducing computational complexity of the resulting algorithm. In addition,
the proposed method utilizes a wrapper sequential forward feature selection
(SFS) and Linear Predictive Coefficients (LPC) for the reduction of features
used for classification. This benefits the system by removing correlated and
irrelevant features, thus reducing computational complexity and removing
irellevant noise from feature vectors prior to training.  Final classifications
are determined through cascaded ensembles of ANNs. The signal is first
classified as either of high or low sound quality, and then as normal or
abnormal. The system achieved a final score of 85.9\% on the hidden test set.\\

Plesinger et al.\ opted to develop a new form of machine learning algorithm
based on probability assesment~\parencite{Plesinger2017}. In this method,
features are mapped to histograms and thought of as probability distributions.
Weights are applied based on the number of occurences of each feature, which in
turn generates a probability function. This can then be used to calculate the
estimated classification of a new data point. From the 228 extracted features,
53 features were then selected based on calculated sensitivity and specificity
scores using generated histograms. This allowed for the training scores to be
automatically optimized by the algorithm.\\

Kay et al.\ present a method using ANNs, a wide variety of features and PCA for
feature reduction~\parencite{Kay2017}. The algorithm scores well on the test
set. However, this work is most notable for it's rigorous evaluation by
authors, using leave-one-out cross validation for a clearer understanding of
the generalisation of the algorithm, as well as highlighting issues with the
underlying database that are discussed in Section~\ref{Database}


\newgeometry{margin=1cm} % modify this if you need even more space
\begin{landscape}
\begin{table}[H]
    \captionof{table}{Summary of top 10 Physionet Challenge 2016 entries}
    \label{PhysionetTable}
\scriptsize
%\centering
\rowcolors{1}{gray!15}{white}
\doublespacing
\begin{tabulary}{\linewidth}{CCCCC}
\dtoprule
Author                                           & Features                                                                        & Classification Method                                    & Reported Scores                                                                                      & Challenge Score \\ \midrule
Potes et.~al \citeyearpar{Potes2016}             & 124 TFR features                                                                & Combined AdaBoost/ANN                                    & In-house test set accuracy: AdaBoost-abstain: 79\%, CNN: 82\%, Combined classifiers: 85\%            & 86.02\%         \\
Zabihi et.~al \citeyearpar{Zabihi2016}           & 40 temporal, spectral and TFR features, reduced using SFS and LPC               & 2 ensembles of neural networks                           & Training accuracy: Maximum of 91.50\%                                                                & 85.90\%         \\
Kay et.~al \citeyearpar{Kay2017}                 & CWT, MFCCs, complexity measures, Inter-beat features, PCA                       & ANNs                                                     & A range of cross validation based tests were used to analyse performance. See paper for full details & 85.20\%         \\
Bobillo \citeyearpar{Bobillo2016}                & MFCCs and WPD, reduced using tensor decomposition                               & $k$-NN                                                   & A range of cross validation based tests were used to analyse performance. See paper for full details & 84.54\%         \\
Homsi et.~al \citeyearpar{Homsi2017}             & 131 time domain, STFT based andwavelet based features                           & Combined ensembles of LogitBoost, Random Forrest and CSC & Training accuracy 87.7\%, In-house test accuracy: 93.24\%                                            & 84.48\%         \\
Maknickas et.~al \citeyearpar{Maknikas2017}      & MFCCs, reduced by Karhunen–Loeve transform                                      & Deep Neural Network                                      & Training accuracy 99.7\%, Validation accuracy 95.2\%                                                 & 84.15\%         \\
Plesinger et.~al \citeyearpar{Plesinger2017}     & Statistical and symettry properties of amplitude envelopes for S1 and S2 sounds & Custom probability assesment machine learning algorithm  & Training accuracy 90.3\%                                                                             & 84.11\%         \\
Rubin et.~al \citeyearpar{Rubin2016}             & MFCCs                                                                           & Convolutional neural networks                            & --                                                                                                   & 83.99\%         \\
Jiayu (paper not submitted)                      & --                                                                              & --                                                       & --                                                                                                   & 82.82\%         \\
Abdollahpur et.~al \citeyearpar{Abdolahpur2017} & time, TFR and perceptual features, reduced using Fisher's discriminant analysis & Combined ANNs                                            & Training accuracy: 91.6\%, 87\%, 84.55\% (prior to ANN combination method)                           & 82.63\%\\
\dbottomrule\\
% TODO: Add footnote explanation for Ac = Accuracy

\end{tabulary}
\end{table}
\end{landscape}
\restoregeometry
% TODO: Summary of the way projects were evaluated in general, and what could be improved
\doublespacing
\section{Database}\label{Database}
%TODO: Briefly describe what is needed from a database for this project
A database representative of real-world PCG signals was needed to train models
and evaluate the proposed method effectively.  A number of criteria were
identified as necessary for the success of the proposed project:
\begin{itemize}
    \item The database must contain sufficient PCG data, so that a model
        trained to discriminate between said signals would, in theory, generalise
        to new PCG data.
    \item The database should contain a mixture of clean and noisy signals that
        represent a variety of real world situations, as real-world
        classification would likely be performed in sub-optimal conditions. If
        this is not possible, noise could potentially be added to clean signals
        to simulate this.
    \item Healthy signals must be able to be differentiated from a variety of
        individual pathologies in order to provide a general abnormality
        detection algorithm. This should be reflected in the database through
        inclusion of a variety of signals representing different pathological
        heart conditions.
    \item Data must be reliably labelled in order to generate a reliable model
        (paticularly when using machine learning methods, as in the proposed
        project). Labels should ideally be verified by a trained professional.
\end{itemize}
\noindent
Two viable options were then considered based on the above criteria:
\begin{enumerate}
    \item The Physionet challenge database
    \item Generation of a synthetic dataset via methods such as that proposed
    by Almasi et al.~\parencite{Almasi2011}
\end{enumerate}

Generation of synthetic data was considered as few well-formed alternative
databases exist, other than the Physionet challenge data. The database curated
for the Physionet challenge was selected for this project, as it fulfilled the
criteria sufficiently and posed less of a risk in terms of signal quality, due
to all signals being produced in real-world environments. However, synthesis
of PCG data remains an interesting possibility for improving evaluation of
classification systems and could be considered for the generation of additional
samples in future work.

\subsection{Database Summary}
The selected database is significantly larger and contains a wider variety of
signal conditions than any database used for previous research (as detailed in
table~\ref{PriorWorkTable}). It is released as an open-source resource and is
documented in significant detail by Liu et al.~\parencite{Liu2016}. The lack
of any alternative databases, comparable in size or variety of content, perhaps
makes this resource the current standard for PCG analysis projects. In
addition, by replicating the conditions of the Physionet challenge, results can
also be directly compared with those of the challenge participant's, with the
aim of understanding how the proposed algorithm compares to the current state
of PCG analysis.

\begin{itemize}
    \item The database consists of 6 sub-databases, labeled $a$ to $f$.
    \item These sub-databases have been sourced from a variety of professionals,
        over the course of a decade.
    \item A total of 3,126 recordings are included, created using varying equipment.
    \item 2575 recordings are labeled as normal, 665 are labeled as abnormal.
    \item All samples have been resampled to 2KHz
    \item Samples were recorded in a range of enviroments, both clinical and
        non-clinical.
    \item Many recordings are corrupted with environmental noise, such as
        microphone friction, breathing, talking etc\ldots
    \item Sections of silence are present in some recordings, most
        significantly in database $e$
\end{itemize}

\subsection{Considerations}\label{DBCons}
There are a number of issues with the acquired database that have been
highlighted, both through previous literature and through development of the
project. These have been considered throughout development and evaluation of
the project.\\
A significant issue highlighted by Liu et al.\ is the large number of normal
recordings compared to pathological recordings. This creates a clear class
imbalance issue that can result in over-inflated classification
results. This is considered in
Section~\ref{Resample}.\\
Another key issue is the difference between the databases used by participants of the
Physionet challenge, and the available data that was acquired for this project.
% TODO: Update to reflect use of quality labels that have now been found
For unknown reasons, information such as patient labels used for training many
of the challenge participant's models have not been made publicly available and
so could not be used in this project.\\
The lack of access to the hidden test set used for evaluating challenge entries
also had a significant impact on evaluation. An alternative method for
evaluating using only the data provided has been proposed in
Section~\ref{Eval}.\\
Finally, an issue is highlighted by Bobillo with regards to database
$e$~\parencite{Bobillo2016}. The recording of normal and pathological signals using
separate devices is likely to cause issues and is discussed in
Section~\ref{Eval}

\section{Design}
This project aims to provide robust heart abnormality detection for PCG
signals, such that use of the system could reliably recommend further medical
attention when neccesary. It is clear from previous research that machine
learning methods for classification have shown the most promise in this area,
and that ensemble methods have been largely sucesful in improving
classification accuracy of base classifiers~\parencite{Homsi2017, Potes2016}.
However, one such method that has recently shown significant success in other
fields, but is not present in recent PCG analysis literature, is the stacking
meta-classifier~\parencite[p.498]{Tobergte2013a}. The presented system was
therefore designed to explore the potential for this classification method in
the context of PCG signal classification. This section details the four key
components developed to form the final system: signal preprocessing
(Section~\ref{preprocessing}), audio feature extraction (Section~\ref{featEx}),
classification (Section~\ref{class}) and optimisation (Section~\ref{optimise}).

% TODO: Create flow diagram Preprocessing -> Feature extraction ->
% Model optimisation -> Performance evaluation

\subsection{Preprocessing}\label{preprocessing}
It quickly became apparent that, due to significant variations in the available
data (as a result of noise, variations in recording equipment etc...), the
effective preprocessing of such data would be a critical factor when designing
the system. This section details the most significant preprocessing steps
taken, in order to both minimize noise and extract the basic structure of the
signal.

\subsubsection{Signal decimation}
A common method employed to simultaneously reduce computation time and remove
extraneous information is to decimate the input signal by an integer factor.
According to shannon sampling theorem, a digital signal can only represent
frequency content up to half the sample rate of the signal (the nyquist
rate)~\parencite[p.140]{Kadis1999}
Therefore, by removing every $n$th sample, high frequency content can be
removed whilst lowering the number of samples that must be processed in
subsequent operations. An anti-aliasing filter must also be applied to the
signal in order to filter harmonic distortion generated by the process.
As it is commonly stated in the literature that little relevant information in
PCG signal is found above 400Hz, all signals were resampled to 1KHz giving a
500Hz cutoff frequency, using an 8th order zero-phase Chebyshev type I filter.

\subsubsection{Dataset resampling}\label{Resample}
A common issue with data collected from the real world is the imbalance of
classes in data. As noted by Liu et al.~\parencite{Liu2016}, this is the case
with the available dataset, as there are less pathological signals than healthy
signals.  This presents an issue with classification tasks, as imbalance can
have a negative impact on classification of the minor class. In this context,
class imbalance could potentially impact classification accuracy for abnormal
samples, so must be handled appropriately. This issue can be approached using a
number of methods. Sophisticated oversampling methods such as SMOTE (Synthetic
Minority oversampling Technique) offer one solution. SMOTE generates synthetic
samples using interpolation and adds these to the data set to balance the
classes, without using direct copies of existing data. However, oversampling
techniques such as this can increase overfitting of models, and don't always
offer reasonable improvement in performance~\parencite{Longadge2013}.
Undersampling is the most common method used, typically by randomly removing
samples from the major class. This has the obvious disadvantage of reducing
data available for training. However, an improved method using $k$-Means
clustering has been shown to be effective in previous cardiovascular
classifications problems~\parencite{Rahman2013}. This method was seen to be the
best choice for the proposed system.

\subsubsection{Signal Segmentation}
%TODO: Generate segmentation plot
With one notable exception~\parencite{Langley2016}, previous classification
algorithms rely heavily on the ability to segment signals into the four
fundamental heart sounds. This is a key prerequisite to the extraction of
relevant features. The defining of signal structure allows for the
relationships between it's components to be analysed, as described in
Section~\ref{featEx}. To faciliatate the development of robust agorithms for
the Physionet challenge, participants were provided with an implementation of
Springer's HSMM based segmentation algorithm. As the highest scoring algorithm
in the literature, it was clearly the most suitable algorithm to use for the
proposed system. In addition to the high accuracy of segmentation, the wide
adoption of this algorithm is beneficial for comparison with other algorithms
submitted to the challenge. Results produced by the proposed system will
generally not be coloured by the differences in quality of segmentation
algorithms, allowing for more direct comparison of classification methods.
However, it is noted that despite the high performance of the algorithm, errors
in segmentation will still occur that may have a negative impact on feature
quality. As methods proposed by previous literature such as hand correction by
a professional~\parencite[p.2203]{Liu2016} are not feasible in this context,
and considering the low number of erroneous results produced by the
algorithm~\parencite[p.2]{Goda2016} it was decided that these errors would not
pose a significant problem.


\subsection{Feature Extraction}\label{featEx}
The extraction of feature vectors from data is a fundamental component of most
machine learning based systems. The aim is to construct meaningful
representations of the data that emphasize information relevant to the
classification problem. In the proposed project, 188 features were extracted
from the data, procuring feature extraction techniques from a wide range of
previous literature, as well as using novel perceptual features commonly found
in audio/music analysis (See Sections~\ref{FFT} and~\ref{Time}).
There are also potential issues that can occur when using large sets of
features for training. The method proposed for addressing these issues is
discussed in section~\ref{SFS}. This section provides a summary of the main
feature categories. Please refer to appendix~\ref{appendixA} for a full
breakdown of all features.

\subsubsection{Time-domain features}\label{Time}
A range of features were generated, based directly on the time series data.
Features such as:
\begin{itemize}
    \item Average and standard-deviation of segment intervals, for all heart
        sounds and complete heart cycles
    \item Ratio of systolic and diastolic period to total heart cycle period
    \item A range of statistical features such as entropy, skewness and variance for
        each heart sound
    \item A selection of envelope based features for each heart sound
\end{itemize}

18 feature provided by the Physionet challenge focused on timings between
segments of the heart cycles. It was thought that these features would be
useful in capturing irregularities caused by conditions such as arrhythmias,
atrial septal defect and other conditions that are likely to affect relative
timing of heart sounds, such as Mitral valve prolapse or regurgitation.
Many conditions that can be detected by traditional auscultation are
characterised by an increase in loudness of the S1 and/or S2 heart
sounds~\parencite{Brown2008}. This suggests that features relating to human
perception of loudness may aid in the detection of such conditions.  Simple
envelope based features such as RMS, peak loudness and the Shannon energy
envelope (Equation~\ref{ShanEQ}, popular in previous literature, were extracted
for this reason~\parencite[p.73-77]{Lerch2012}. In addition, statistical
features such as sample entropy and skewness (Equation ~\ref{SkewEQ}) were used
to evaluate the distribution of samples for each heart sound, these were
selected to provide a representation of the temporal ``shape'' of each sound.

\begin{equation}\label{ShanEQ}
    SE = \frac{-1}{N}\sum\limits_{n=0}^N x(n)^2\cdot \log{x(n)^2}
\end{equation}
\begin{equation}\label{SkewEQ}
    S=\frac{E(x-\mu)^3}{\sigma^3}
\end{equation}
Where:\\
$x(n)$ is the input signal\\
$E(t)$ is the expected value\\
$\mu$ is the mean of the signal\\
$\sigma^2$ is the variance of the signal

\subsubsection{FFT-based features}\label{FFT}
It was recognised that a time domain representation alone was unlikely to
provide a sufficient representation for discerning a wide variety of
conditions. Using a time-frequency representation to characterise the spectral
components of the signal has proven effective in the majority of literature.
The classic method for producing a spectral representation of a signal is the
Fourier transform (as defined in Equation~\ref{FFTEQ}) over a sliding window of size
$N$. By decomposing the signal into a series of sine and cosine
waves, a representation of the signal across a range of frequency bands is
produced. This can be used for further analysis of heart sounds
based on their spectral characteristics.
\begin{equation}\label{FFTEQ}
X(k)=\sum\limits_{n=0}^{N}x(n)e^{\frac{-j2\pi kn}{N}}
\end{equation}
Where $x(n)$ is the input signal\\
Features generated using this representation would, in theory, be useful for
identifying conditions that reside in specific frequency bands, such as
murmurs, for example~\parencite{Sepehri2010}.\\

An example of such features are Mel-Frequency Cepstrum Coefficients (MFCCs).
Popular in speech processing, MFCCs provide a compact representation of a
signal's spectral shape. MFCCs are calculated by first applying $N$ (a
user-defined parameter) triangular filter banks, spaced using the mel scale to
the magnitude spectrum. Applying a discrete cosine transform to the log of the
filterbank outputs provides the final set of coefficients (for further details,
please refer to~\parencite{Lerch2012}). This representation
creates a perceptually relevant representation of spectral shape, in effect
mimicking the way in which humans might perceive the spectral shape of heart
sounds. The reasoning for this is that, as the aim is to provide a system with
performance better than, or equal to to that of a human, features that mimick
what a human percieves may prove effective at distinguishing conditions in the
way that a human does. This has shown to be effective in previous literature,
with multiple systems utilising perceptual features with
success~\parencite{Ortiz2016, Rubin2016, Quiceno-Manrique2010a}. 13 MFCCs were
calculated for each heart sound and averaged per sample to provide 13 features
per sample.\\
%TODO: Generate MFCC spectum

In addition to MFCCs, other statistical features were extracted from the
spectrum such as spread, skewness, kurtosis and flatness. These features aim to
provide alternate spectral measurements to MFCCs, in a similar way to their
temporal counterparts as described in Section~\ref{Time}.

Although the Fourier representation of PCG signals has proven effective in many
cases, there are drawbacks of this representation that must be considered. One
key issue that is inherent of fourier transforms is the time-frequency
tradeoff. An increase in frequency resolution will always result in a decrease
in temporal resolution. This poses a problem, as it is not possible to localize
transient events accurately in the frequency domain using this method. This
method may also suffer in the presence of background noise common in PCG
signals. Previous studies have shown that these factors may have a significant
impact when detecting conditions such as Coronary
Stenoses~\parencite{Ergen2001, Akay1990}

\subsubsection{Wavelet decomposition features}
The wavelet transform has been used effectively as an alternative
time-frequency representation to fourier methods. The fundamental concept of
the wavelet transform is to represent an input signal as a set of scaled and
shifted finite oscillations. By comparing the signal with each scale of wavelet
at all points in time, a set of $N\times A$ (Where $A$ is the number of scales)
coefficients are generated that represent the scale and position needed for
each wavelet in order to fully reconstruct the signal (For further details,
refer to~\parencite{Polikar1994}) The benefit of this transform is that it is
well localized in both time and frequency. This allows for accurate
representation of transient events such as clicks and snaps that are
characteristic of heart conditions such as Mitral valve prolapse or
stenosis~\parencite{Brown2008}.\\
For the proposed system, a 5 level DWT using debauchies-4 mother wavelet was
used for decomposition and reconstruction. Statistical features such as entropy
were then calculated, both on the reconstructed signal and directly on
coefficients to attain a total of 48 features.~\parencite{Homsi2016}
% TODO: Insert wavelet diagram here

\subsubsection{Feature Scaling and Imputing}
A common problem when working with multiple features is the difference in scale
Dbetween features. This problem can cause many machine learning algorithms to place
bias on larger scale features and can significantly impact the time taken for
certain algorithms to converge. This is particularly significant when applying
algorithms sensitive to feature scale such as SVMs (described in
Section~\ref{SVM}). To address this, a Min-Max scaler was applied
to training and test sets prior to training models. This scales all values to within a
0--1 range producing a set of features on a common scale.\\
It is also common to encounter missing values in features. these can occur as a
result of $\log(0)$ or division by 0 calculations, amongst other edge cases. A
standard method for handeling these values is to apply an imputer, replacing
values with the mean of the feature vector.~\parencite{VanderPlas2017}

\subsection{Stacking Classifier with Cross-Validation}\label{class}
The stacking classifier is an ensemble classifier, that uses the results of
multiple base classifiers as input to a 2nd level meta-classifier, used to
generate a final predicition. $k$-fold cross validation is used accross base
classifiers, training on $k-1$ folds of input data, and applying to the
remaining hold out set. The results of these predictions from each base
classifier are combined and used to train the 2nd level classifier which
produces the final preditions.\\
Given it's considerable performance accross a range of tasks, it was expected
that this classification model could be applied effectively to produce an
alternative method for abnormality detection than those presented in previous
literature.
% TODO:Insert stacking classifier diagram

\subsubsection{Base Classifiers}
Clearly, an important consideration when using any ensemble method is the
selection of the base classifiers. In order for any ensemble method to perform
well, it must be constructed using a selection of classifiers that individually
provide useful models for the data~\parencite[p.484]{Tobergte2013a}.  The final
optimized model consisted of 3 base models. A wide variety of models were
considered for use as base and meta models. These included models such as Tree
based, $k$-Nearest Neighbor, and AdaBoost classifiers. Selection of these
models was based on a novel approach using hyperparameter optimization as
discussed in Section~\ref{optimise}. The following sections detail the final
selection used; A combination of SVM and Naive-Bayes classifiers, with a
Logistic Regression meta classifier.

\paragraph{SVM}\label{SVM}
The SVM classifier aims to fit a hyperplane to data that maximises the
separability between classes. This results in a model that has been shown to
generalise well in many cases, as maximising separability between classes is
also likely to increase the margin for error in separation of classes. This
type of classifier is also able to generate hyperplanes in non-linear space,
using a techniques known as `kernal tricks'. This works by mapping linear data
to a higher dimension, allowing non-linearly seperable classes to be separated
by the same method. The details of the SVM and Kernal-SVM are involved and
outside the scope of this report. Further details can be found
in~\parencite[p.187]{Tobergte2013a}.\\
% TODO: Create Hyperplane plot
SVMs have been prevalent in previous literature, shown to be effective in
separation of a variety of heart conditions~\parencite{Ari2010} The use of
kernals to map parameters to higher dimensions is a key advantage of this
model, allowing for non-linear relationships that are likely to be present in
the large variety of features to be well represented in classification. Choice
of kernals, and relevant hyperparameters is detailed in Section~\ref{optimise}.

\paragraph{Naive-Bayes}
Commonly used in text classification problems, where there is typically a
high-dimensional feature space, Naive Bayes classification uses Bayes rule to
determine the probability of classification, given a vector of features. This
is calculated as:
\begin{equation}
    P(y\mid x_1,\ldots,x_n)=\frac{P(y)\prod\limits_{i=1}^{N}P(x_i\mid y)}{P(x_1,\ldots,x_n)}
\end{equation}
The implementation used assumes a gaussian distribution for all features,
calculating the probability of a feature as:
\begin{equation}
    P(x_i\mid y)=\frac{1}{\sqrt{2\pi
    \sigma_y^2}}\exp\bigg(-\frac{(x_i-\mu_y)^2}{2\sigma^2_y}\bigg)
\end{equation}
Where:
$\mu$ is the mean of the distribution
$\sigma^2$ is the varaince
Using Maximum Liklihood estimation to estimate $\sigma$ and $\mu$ given the
feature vector, a classification for new features can then be calculated as:
\begin{equation}
    \hat{y}=\argmax\limits_y P(y)\prod\limits_{i=1}^nP(x_i\mid y)
\end{equation}
Where:\\
$x$ is the feature vector to be classified\\
$\hat{y}$ is the estimated classification\\

Despite their computational simplicity, Naive Bayes classifiers have been shown
to produce highly accurate classifications models. The assumption that each feature is
completely independant allows for extremely fast classification and scalability
to large datasets, with many dimensions~\parencite[p.300]{Zhang2004}. It was
thought that these benefits would make the classifier suitable for the proposed system, as the reatively high
dimensionality of features and quantity of datapoints could then be classified
quickly to obtain initial results. Despite the inclussion of more complex
models, this model was chosen via automatic selection for the final model.
Refer to section~\ref{PSOp} for further details.

\paragraph{Logistic Regression}
Logistic regression is a regression model that aims to fit as hyperplane to
data points by minimizing a cost function using weighted features.
By applying weights to feature vectors then applying a sigmoid function, a
hypothesis function is defined as:
\begin{equation}
    h_\theta(x)=\frac{1}{1-e^{-\theta^{T}x}}
\end{equation}
Where:\\
$x$ is a feature vector\\
$y$ is a class label vector \\
$\theta$ is a weight vector \\
A cost function can then be defined as:
\begin{equation}
    J(\theta)=\argmin\limits_\theta\frac{1}{2m}\sum\limits_{i=1}^m\Big(h_\theta(x^{(i)})-y^{(i)}\Big)^2+\text{Regularization}(\theta)
\end{equation}

\begin{align}
    &\text{Regularization}{(\theta)}_\text{L1}=\lambda\sum\limits_{j=1}^n\mid\theta_i\mid\\
    &\text{Regularization}{(\theta)}_\text{L2}=\lambda\sum\limits_{j=1}^n\theta_i^2
\end{align}
Where:
$\lambda$ is the regularization parameter used to help prevent overfitting\\
By minimizing the cost function, classification predictions can then be made
using the hypothesis function~\parencite{Ng2012}.\\
Logistic regression was chosen as the meta-classifier primarily due to it's
simplicity and performance in testing. Choice of meta-classifier is a potential
area for improvement and it is noted that a range of meta-classifiers have been
proposed for different tasks that utilise
stacking~\parencite[p.29]{Sesmero2015}. Further work in this area could
potentially provide improved results.

% TODO: Replace this section
% \subsubsection{Signal quality classification}\label{Quality}

\subsection{Model Optimization}\label{optimise}
As discussed in previous section, two of the most important aspects that affect
the performance of a classification system are it's models, and the input
features. A combination of relevant features and well tuned models is therefore
likely to provide an accurate classification system. However, it is not always
immdiately clear which values to choose for parameters, or features to use as
input. This is especially true when given such a wide selection of models to
choose from, and high such dimensional feature spaces, as are used in the
proposed method. To address this issue, two automatic optimisation approaches
were implemented with the aim of maximising the accuracy of the proposed
system. 

\subsubsection{Sequential Feature Selection}\label{SFS}
It was recognised that the extraction of such large numbers of features in the
proposed system would likely result in a large amount of redundent information.
There are two commonly used methods for addressing this problem: feature
reduction and feature selection. Feature reduction involves reducing features
to a lower dimensionality using techniques such as PCA. Conversely, feature
selection involves selectively removing features entirely via methods such as
Sequential Floating Selection (SFFS). Both aim to reduce the amount of redundant
information in features by removing or reducing features that are expected not
to benefit the model. As a selection of models were to be used, each
potentially handeling dimensionality differently (SVMs in particular), it was
decided that feature selection would be most appropriate for this application.\\

Through experimentation, the chosen method was SFFS. This method is an adaption
of tradition sequential forward selection, that also uses sequential backward
selection to allow for subsequent removal of added features when neccesary.
SFFS is an iterative wrapper method that adds features and retrains the chosen
model sequentially, choosing features that increase the accuracy of the model
output (using 3-fold cross validation to avoid overfitting). Final models used
as few as 40 features, increasing both accuracy of classifications and
computation time of models significantly. For further details on SFFS please
refer to~\parencite[p.3]{Ferri1994}

\subsubsection{Particle Swarm Hyperparameter Optimisation}\label{PSOp}
The particle swarm optimization algorithm is an iterative meta-heuristic algorithm that
aims to find the set of parameters that maximises a given function. Given a
$n$ dimensional parameter space, the algorithm randomly initialises sets of
`particles' representing random combinations of parameters. As the algorithm
progresses particle travel through the parameter space, updating their
position based on their velocity, best historical score and the best historical
score of the swarm. As the algorithm iterates, particles will converge on local
optima, producing potential solutions. The best score is chosen after the final
iteration as the best parameter selection. Annotated pseudocode for this
algorithm is shown in code block~\ref{PSCode}~\parencite{Clerc2002}

\onehalfspacing
\begin{lstlisting}[escapeinside={(*}{*)}, label={PSCode}, caption={Particle
Swarm Optimization Pseudocode}]
Do
    //For all particles...
    For (*$i$*)=1 to Population Size
        // If the current function score is better than the historical
        // function score for the current particle, store new position
        if (*$f(\overrightarrow{x}_i)>f(\overrightarrow{p}_i)\text{ then }\overrightarrow{p}_i = \overrightarrow{x}_i$*)
            // Store best position of all particles in neighbourhood
            (*$\overrightarrow{p}_g=\text{max}(\overrightarrow{p}_{\text{neighbors}})$*)
                // For each dimension in the parameter space...
                For (*$d=1$*) to Dimension
                    // Update velocities
                    (*$v_{id}=v_{id}+\phi_1(p_{id}-x_{id}+\phi_2(p_{gd}-x_{id})$*)
                    // Ensure particle velocity is within limit
                    (*$v_{id}=\text{sign}(v_{id})\cdot \text{min}(\text{abs}(v)_{id}, v_{\text{max}}))$*)
                    // Update particle position
                    (*$x_{id}=x_{id}+v_{id}$*)
                Next (*$d$*)
    Next (*$i$*)
Until termination criterion is met
\end{lstlisting}
\doublespacing

The use of this algorithm allowed for the efficient optimisation of all parameters
relating to the stacking classifier, and it's base classifiers, resulting in a
finely tuned classification model that would not have been produceable using
traditional trial and error methods.\\
During the initial design phase, it was found that the abundance of machine
learning algorithms available make selection of the optimal model a difficult,
requiring in depth knowlege of a range of machine learning techniques. A novel
approach used by recent stacking classifier applications has been in the use of
meta-heuristic algorithm to select models automatically, in addition to tuning
parameters. By thinking of the base classifiers as hyperparameters themselves,
models can be swapped in and tuned automatically by the particle swarm
algorithm to provide a locally optimal selection of base classifiers for the
model~\parencite{Sesmero2015}. This technique was used to pick the 3 final
models described in section~\ref{class} from a selection of 8 models. This
dynamic selection of models was seen to be one of the key contributors to the
overall success of the agorithm.

\subsection{Model Performance Evaluation Method}\label{metrics}
In order to fully understand the performance of the system (and to evaluate the
impact of design decisions throughout development), a group of scoring methods
were implemented to test the system's performance in a selection of scenarios.
The aim was to provide reliable metrics that would highlight the systems
strength and weaknesses and to provide quantifyable measures with which to
compare the system to the range of alternative methods proposed in the
literature.\\

One of the most basic metrics was the scoring of the trained model on a
separate hold-out dataset. By reserving a selection of samples from accross the
databases, a trained model could then be scored on this dataset for accuracy, sensitivity and
specifcity (metrics described in Section~\ref{ChallengeEnt}) to determine the
system's performance on an unseen set of samples. This method is widely used to
provide a basic understanding of a model's ability to generalise to new data, A
crucial requirement of the system. Data was split using a grouped stratified shuffle
split, grouping by database. This ensured an equal number of randomly selected
classes were taken from each database to produce training and test sets. This
approach was taken to avoid class imbalance issues caused when there is a
significant difference in class frequency, as detailed in
Section~\ref{Resample}. It should be noted that although samples were
stratified by class, it was not possible to stratify samples by patient. This
may have an impact on results, as the presence of data from the same patient in
both training and test set may artificially inflate results, where the model has
learnt patterns specific to that patient that do not generalise to others.\\

A more robust method for for model evaluation is cross-validation. By splitting
the full dataset into multiple folds, and training models on each, metrics can
be calculated on each fold, and an average can be taken to provide a measure of
the system's performance over all folds. 10-fold cross validation, stratified
by class, was chosen for evaluation of the system. This provides an insight
into the performance of the algorithm accross the dataset.\\
It is highlighted by Homsi et.\ al, that a large amount of variance may be observed
accross folds~\parencite[p.1637]{Homsi2017}. Homsi et.\ al attribute this to the
variations accross databases, making generalisation difficult. To account for
this, it is suggested that cross-validation is repeated multiple times and
average to provide a more accurate measurement of performance accross folds.
For the proposed system, cross validation was repreated 10 times for each fold
and averaged to produce the final results. Standard-deviation is also
calculated accross these iterations to illustrate the possible prevelance of
this.\\

A common theme throughout the literature was that of generalisation accross
databases. It was observed that many previous algorithms achieved high
accuracies in standard cross-validation, but performed significantly worse when
testing on unseen databases~\parencite{Homsi2017, Bobillo2016}. For this
reason, leave-one-out cross-validation was used to form a better understanding
of the system's ability to generalise to unseen data from different sources. On
each fold, a single database is removed, training on all other databases.\\

The evaluation of models using cross-validation was not limited to final
evaluation. Evaluation of intermediate models generated by both the SFFS and Particle Swarm
algorithms was possible, by further separating the test set into test and
validation fold. This technique was used in the optimisation of the
final model to provide scores for intermediate models, as well as to help avoid
overfitting parameters and features to the training set used for
optimisation.\\

Discussion on the performance of the proposed system using these methods can be
found in Section~\ref{Eval}.
% TODO: Insert cross validation diagram from data science handbook

\section{Implementation}
This section describes the tools used in the realisation of the
proposed system and the practical issues encountered throught the
implementation process. Rationale is given for decisions made throughout
production of the proposed system and any issues with curent implementation are
outlined.

\subsection{System Structure}
From the outset, the project aimed to 

focus on using open source libraries throughout the project to avoid
`reinventing the wheel'. Integration of external libraries
Use of Python - quick development, wide variet of third party libraries to
allow for rapid prototyping


Interface
- Implementation of simple CLI for quick control of system parameters
- High computational cost - Multiprocessing, logging issues
Data Manipulation
- Pandas and Numpy for basic handeling and manipulation of data
- Splitting of data using sklearn
Implementation of features
- Joining of existing segmentation script and python code
- pyWavelets for wavelet features
- librosa for MFCCs
Implementation of machine learning classifiers
- Use of sklearn for base classifiers, use of pipelines
- Addition of stacking classifier using mlxtend - use of probabilities
- Saving of features and models to pickles, allowing for direct running of
intermediate section of system and for development and portability of generated models
Implementation of optimisatons
- Optunity for Hyperparameter optimization
- Mlxtend for SFS




\section{Evaluation}\label{Eval}
Weighted specificity and weighted Accuracy measures
Computational cost was not considered, unlike other entries to the physionet
challenge
Could be used as cloud based system
Discussion on reasons for final selection of models
Features were selected for their individual relevance to classification
problem, Naive Bayes treats features individually. Could explain why it
performed well
Relationships between features likely with features such as wavelets, perhaps
captured by SVMs
Discuss issues with database e
\section{Further Work}\label{FurtherWork}
Handle silent sections of audio such as those highlighted by Goda et.\
al~\parencite{Goda2016}
Synthesis of synthetic PCG signals
Particle swarm Would ideally be placed inside feature selection
% TODO: Consider talking about resampling using Homsi2016 method

\appendix
\section*{Appendices}
\addcontentsline{toc}{section}{Appendices}
\renewcommand{\thesubsection}{\Alph{subsection}}
\subsection{Table of Features}\label{appendixA}
\subsection{Commandline Interface}
\singlespacing
\lstset{basicstyle=\scriptsize, style=mystyle}
\begin{lstlisting}[numbers=none]
usage: main.py [-h] [--features-fname OUTFNAME] [--segment] [--optimize]
               [--eval EVAL] [--select-features SELECT_FEATURES] [--backward]
               [--parameters_fname OUTFNAME] [--fs_fname OUTFNAME]
               [--no-parallel] [--reanalyse] [--verbose]
               [--resample-mix RESAMPLE_MIX] [--keep-logs]
               TESTDIR OUTDIR

Script for the classification of PCG data.

positional arguments:
  TESTDIR               Directory of test data to train the system
  OUTDIR                Directory to store output analyses

optional arguments:
  -h, --help            show this help message and exit
  --features-fname OUTFNAME, -o OUTFNAME
                        Specify the name of the file to save generated
                        features to for future use
  --segment             Run Matlab segmentation script to create segmentation
                        analysis
  --optimize            Run optimization algorithm to find best model and
                        parameters for classifier
  --eval EVAL, -e EVAL  Number of evaluation to pass to the particle swarm
                        optimization
  --select-features SELECT_FEATURES
                        Run feature selection algorithm to find best features
                        for model, either selecting or reducing features by
                        the integer specified. This depends on use of
                        --backward flag, to determine forward or backward
                        feature selection. (a value of 0 skips feature
                        selection entirely, using previously generated
                        features if available. A value less than 0 uses all
                        available features.)
  --backward, -b        Runs backward feature selection as opposed to default
                        forward selection.
  --parameters_fname OUTFNAME
                        Specify the name of the file to save generated
                        features to for future use
  --fs_fname OUTFNAME   Specify the name of the file to save generated feature
                        selection model to for future use
  --no-parallel, -p     Disable processing in parallel. (Will likely decrease
                        performance but may aid in debugging)
  --reanalyse           Force regeneration of database features
  --verbose, -v         Specifies level of verbosity in output. For example:
                        '-vvvvv' will output all information. '-v' will output
                        minimal information.
  --resample-mix RESAMPLE_MIX, -r RESAMPLE_MIX
                        Mix between bootstrap and jacknife resampling used to
                        balance the dataset (0=just jacknife, 1=just bootsrap)
  --keep-logs           Keep previously generated logs that aren't overwritten
                        by current process
\end{lstlisting}
\doublespacing


\pagebreak{}
\printbibliography{}

\end{document}
