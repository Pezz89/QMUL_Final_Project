\documentclass[titlepage, 12pt]{scrartcl} \usepackage{enumitem}
\usepackage[british]{babel}
\usepackage[style=apa, backend=biber]{biblatex}
\DeclareLanguageMapping{british}{british-apa}
\usepackage{url}
\usepackage{float}
\usepackage[labelformat=empty]{caption}
\restylefloat{table}
\usepackage{perpage}
\MakePerPage{footnote}
\usepackage{abstract}
\usepackage{graphicx}
\usepackage{setspace}
% Create hyperlinks in bibliography
\usepackage{hyperref}
\usepackage{amsmath}

\usepackage[pass]{geometry}
\usepackage{graphicx}

\usepackage[T1]{fontenc}
\usepackage[utf8]{inputenc}
\usepackage{blindtext}
\setkomafont{disposition}{\normalfont\bfseries}

\usepackage{etoolbox}
\graphicspath{{./resources/}}
\addbibresource{~/Documents/library.bib}

%\newsavebox{\abstractbox}
%\renewenvironment{abstract}
%  {\begin{lrbox}{0}\begin{minipage}{\textwidth}
%   \begin{center}\normalfont\sectfont\abstractname\end{center}\quotation}
%  {\endquotation\end{minipage}\end{lrbox}%
%   \global\setbox\abstractbox=\box0 }

%\makeatletter
%\expandafter\patchcmd\csname\string\maketitle\endcsname
%  {\vskip\z@\@plus3fill}
%  {\vskip\z@\@plus2fill\box\abstractbox\vskip\z@\@plus1fill}
%  {}{}
%\makeatother

\DeclareCiteCommand{\citeyearpar}
    {}
    {\mkbibparens{\bibhyperref{\printdate}}}
    {\multicitedelim}
    {}

% MATLAB Code block stuff...
\usepackage{color}
\usepackage{listings}

\definecolor{dkgreen}{rgb}{0,0.6,0}
\definecolor{gray}{rgb}{0.5,0.5,0.5}

\lstset{language=Matlab,
   keywords={break,case,catch,continue,else,elseif,end,for,function,
      global,if,otherwise,persistent,return,switch,try,while},
   basicstyle=\ttfamily,
   keywordstyle=\color{blue},
   commentstyle=\color{gray},
   stringstyle=\color{dkgreen},
   numbers=left,
   numberstyle=\tiny\color{gray},
   stepnumber=1,
   numbersep=10pt,
   backgroundcolor=\color{white},
   tabsize=4,
   showspaces=false,
   showstringspaces=false}


\begin{document}
\newgeometry{lmargin=1.5cm}
\begin{titlepage}

    \begingroup

    \setlength{\tabcolsep}{1.5cm}
    
    \begin{tabular}[c]{p{0.30\textwidth} | p{0.4\textwidth}}
    
    {\vspace{1.2cm} \Large School of Electronic Engineering and Computer Science \par} 
    & 
    {\vspace{1.2cm} \large Sound and Music Computing \newline Project Report \the\year \par}\\
    
    & {\vspace{0.5cm} \Large \textbf{Extraction of Statistical Features from PCG Signals for the
Classification of Heart Abnormalities} \par}\\
    
    \vspace{0.4\textheight}
    \includegraphics[width=5cm]{qmul_logo}
    &
    {\vspace{1cm} \large \textbf{Samuel Perry}}\\
    
    &
    \multicolumn{1}{|r}{August \the\year}
    
    \end{tabular}

    \endgroup

\end{titlepage}
\restoregeometry

\doublespacing
\begin{abstract}
   Things and stuff and words...
\end{abstract}

\renewcommand{\abstractname}{Acknowledgements}
\begin{abstract}
I'd like to thanks anyone and everyone...
\end{abstract}

\tableofcontents
\newpage

\section{Related Work}
There are currently a wide variety of methods employed for the analysis and
classification of PCG signals. Current research can be divided into 3 areas,
each of which are combined to create full classification system. These areas
are: signal preprocessing and segmentation, feature extraction methods and
classification methods.

\subsection{Signal Preprocessing and Segmentation}
Due to factors such as recording conditions and 

Algorithms for the pre-processing and segmentation of PCG data
aim to  extract the structure of the signal over time. This is a key
stage in the analysis of PCG signals as the structure and relationships between the
fundamental heart sounds (FHSs) form the basis for much of the further
analysis performed on PCG data. A number of methods exist for the
extraction of FHSs. Some rely on direct extraction of peaks in the time
domain to determine the structure of a signal. These methods perform
various transformation in order to accentuate the transient events with
the intention of isolating them~\parencite{Groch1992, Liang1997}.
However, these methods tend to suffer significantly from background
noise and so perform poorly in sub-optimal conditions.\\
Other methods rely on spectral representations to assist in the
splitting of the FHSs, in particular using wavelet
decomposition~\parencite{LiangHuiying1997, Vepa2008}. This allows for
the separation of components based on their frequency content in place
of, or in addition to their temporal characteristics.\\
In addition, Machine learning algorithms have been employed, such as
$k$-Nearest Neighbour~\parencite{Gupta2007} and Neural
Networks~\parencite{Oskiper2002} to improve segment classification.
More recently, particular success has been observed in Springer's use
of logistic regression and Hidden semi-Markov
models~\citeyearpar{Springer2016}.

\subsection{Statistical Feature Extraction}
A wide variety of methods exist for the extraction of statistical
features from PCG data. These features are used for the creation of
robust, meaningful representations of the data.\\
The use of spectral representations for PCG data are prominent in the
literature. The ability to separate activity across the frequency
spectrum reveals patterns that may not be attainable by analysing the
time domain signal alone.\\
Due to the need for low frequency analysis and the high noise levels
found in PCG signals, it has been found that the traditional FFT
method for extracting spectral information may not be
suitable~\parencite{Akay1990}. For this reason, parametric methods for
spectral estimation have been a popular choice for extraction of such information.
Methods such as AR, ARMA, AR-HOS and MUSIC have been shown to provide spectral
representations suitable for analysis and classification of heart
sound~\parencite{Ergen2001, Schmidt2015}.\\
Other methods such as Wavelet Decomposition and MFCCs have also been
successfully  employed for extracting spectral data for purposes such
as heart valve disease identification and heart murmur
detection~\parencite{Quiceno-Manrique2010a, Maglogiannis2009}.\\

In addition to direct analysis on the signal, the ability to segment
and extract RR values from the signal allows for their statistical
analysis, both in the time and frequency domain, for use as features.\\
Dash et al.\ use a number of time-based statistical analysis on the RR
time series for the detection of atrial fibrillation. Statistical
analyses such as RMSSD, Shannon Entropy and Turning-point Ratio are
used as feature vectors for classification of
signals~\citeyearpar{Dash2009}.  A similar approach is used by Yaghouby
et al.\ for the generalized classification of heart abnormality. Here,
a selection of linear and non-linear features are used for
classification with promising results~\citeyearpar{Yaghouby2009}.\\
Frequency domain analysis of RR values are also used by calculating the
PSD of the RR values via  approaches such as VFCDM.\ This form of
approach allows for higher resolution time-frequency representations of
the RR data than approaches such as the FFT or wavelet transform~\parencite{Wang2006}.
From a spectral representations such as this, Yaghouby et al.\
demonstrate the use of such descriptors for the discrimination between
sympathetic and parasympathetic contents of the signal, not directly
detectable through time domain analysis~\citeyearpar{Yaghouby2009}.\\
Further in-depth analysis of statistical features for HRV can be found
in~\parencite{Electrophysiology1996}

\subsection{Signal Classification}
Classification of signals for diagnostic purposes.  The aim being to
distinguish healthy signals from those with certain heart
conditions/abnormality. This is most commonly achieved by extracting
sets of features vectors from PCG signals, followed by their
classification, most commonly using machine learning algorithms for
automatic classification. The features extracted and classification
algorithms applied vary across the literature based on factors such as
the diagnostic aims of the classification and computing performance
requirements.\\

Artificial neural networks and support vector machines have proven to
be popular choices for classification. Much success has been seen in
employing these machine learning techniques for classification across
both PCG and ECG data for conditions such as chronic heart failure,
atrial fibrillation and flutter, diastolic murmurs, and for general
pathology detection~\parencite{Cathers1995, Wu1995, Bung2000,
Lubaib2016, Maji2014, Ari2010, Maglogiannis2009}. Results do vary based
on the combination of features and exact classification methods used.
However, encouraging results are presented with highly accurate
classifications for general abnormality detection and for more specific
pathological condition detection.\\

However, there is a lack of research into other machine learning
techniques such as bayesian classification~\parencite{Lubaib2016},
$k$-Nearest Neighbour~\parencite{Quiceno-Manrique2010a, Lubaib2016} and
Linear Regression~\parencite{Orhan2013}. Studies that utilize these
methods for classification have generated promising results.  There is
therefore the potential for further research into exploiting the
benefits of these techniques for heart abnormality detection.\\

The selection of features used for classification also depends
predominantly on the aims for the classification. For general
abnormality classification, spectral representations such as wavelet
transformations, VFCMD, FFTs and MFCCs are a popular
choice~\parencite{Bung2000, Wu1995, Yaghouby2009, Dash2009}. Their
multi-dimensional representation of the data reveals details in the
signal that cannot be seen through a 1 dimensional time series alone,
allowing for more accurate classification. Higher-level statistical
methods are also widely used for both time and spectral
representations~\parencite{Bung2000, Quiceno-Manrique2010a,
Schmidt2015, Dash2009, Yaghouby2009}. These allow for the
classification based on more specific statistical properties of the
data. It is highlighted by Orhan that Higher level statistical methods
may add considerable complexity to computations, and so care should be
taken, particularly when considering systems in a real-time
context~\citeyearpar{Orhan2013}.

\section{Dataset}

\section{Design}
The system aims to provide robust heart abnormality detection for PCG signals,
such that use of the system could reliably recommend further medical attention
when neccesary. 
\subsection{Signal Segmentation}
\subsection{Choice of features}
\subsection{Feature selection method}
dimensionality reduction
\subsection{Classification Algorithm}

\section{Implementation}
\section{Evaluation}
Group cross-validation
Weighted specificity and weighted Accuracy measures
\section{Conclusion}



\pagebreak{}
\printbibliography{}

\end{document}
